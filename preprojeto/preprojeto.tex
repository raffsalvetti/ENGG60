\documentclass[a4paper,12pt]{article}
\usepackage{times}
\usepackage[utf8]{inputenc}
\usepackage[brazil]{babel}
\usepackage[T1]{fontenc}
\usepackage{cite}
\usepackage{amsmath,amsfonts,amssymb} % pacote matematico
\usepackage{array}
\usepackage{graphicx} % pacote grafico

\usepackage{psfrag}
\usepackage{pstricks}
\usepackage{epsfig}
\usepackage{url}           % Legenda no texto e na Lista de figuras diferentes. Habilita \cite{} na legenda.

\setlength{\parindent}{1.2cm}
\setlength{\parskip}{.2cm}
\setlength{\oddsidemargin}{0.5cm}    % Há um offset obrigatorio de
\setlength{\evensidemargin}{0.0cm}   % 1 inch do lado esquerdo e no
\setlength{\topmargin}{-1.2cm}       % topo da folha
\setlength{\headsep}{1.0cm}
\setlength{\textwidth}{15.5cm}
\setlength{\textheight}{24.2cm}
\renewcommand{\baselinestretch}{1.2}
\renewcommand{\labelitemi}{\tiny{\textbullet}}  % Define o ícone principal do itemize

%\hyphenation{Trans-ac-tions}


\begin{document}
\pagenumbering{roman}

\begin{titlepage}
\thispagestyle{empty}
\begin{center}
\large{\bf{UFBA - Universidade Federal da Bahia}} \\
\large{\bf{Graduação em Engenharia da Computação}} \\
\end{center}
\vfill

\centering
\textbf{{\LARGE PLANO DE TRABALHO}}  \\ \vspace{0.5cm}
{\LARGE Graduação}
\vfill




\textbf{{\Large Robô de inspeção com visão imersiva}} \\ %\vspace{0.3cm}
%\hrulefill \\
\vfill
\begin{flushleft}
\textbf{Aluno}: Raffaello Salvetti Santos \hfill{}\\
\textbf{Orientador}: --- \hfill{}\\
\textbf{Linha de Pesquisa}: Robótica {\it ou} Sistemas Robóticos \hfill{}\\
\end{flushleft}

\vfill


\vspace{\stretch{1}}
\begin{center}
\large{\bf{\today}}
\end{center}
\end{titlepage}


\pagenumbering{arabic}

\begin{abstract}
Inspeção de áreas de difícil acesso ou que apresentam perigo, como dutos de ventilação, linhas de distribuição de energia eletrica e reservátorios de produtos quimicos corrosivos, é uma realidade na indústria. O uso de robos operados remotamente é uma solução que oferece segurança ao operador. O objetivo deste trabalho é desenvolver de um robo remotamente controlado visando eficiencia, versatilidade e baixo custo de produção.
\end{abstract}


%---------------------------------------------------------------------------------------------------------
\section{Introdução ao Tema}

Faça uma pesquisa bibliográfica prévia para mostrar o historico do problema, o que já foi pesquisado e onde estão as lacunas para investigação. Demonstre o estado da arte do problema e até onde as pesquisas recentes evoluíram o tema proposto. Introduza sua provável contribuição a ser obtida ao término do projeto de pesquisa de mestrado.


%---------------------------------------------------------------------------------------------------------
\section{Justificativa}
Justificativa da razão da escolha do tema, sua relevancia, viabilidade e integração com as linhas de pesquisa do PPGSE.



%----------------------------------------------------------------------------------------------------------------------
\section{Objetivos}
Descreva o objetivo geral de sua proposta de pesquisa. Evidencie, através de objetivos específicos, o que deve ser investigado/trabalhado para se atingir o objetivo geral.


%---------------------------------------------------------------------------------------------------------
\section{Metodologia}
Descreva como você pretende desenvolver o trabalho. Serão utilizadas simulaçoes? Será implantado um prototipo para validação experimental? Será uma análise matemática mais formal? Como você espera alcançar os resultados da sua dissertação? Você deve escrever como será o procedimento para conduzir seu trabalho de pesquisa. Procure destacar os prováveis materiais/equipamentos necessários.


%---------------------------------------------------------------------------------------------------------
\section{Resultados Esperados}
Explicite qual será a utilidade da pesquisa, a quem deverá importar os resultados, o que será produzido e o que se espera, enfim, com a elaboração do seu trabalho. 


%---------------------------------------------------------------------------------------------------------
\section*{Referências Bibliográficas}
Todas as referências desta seção devem ser citadas ao longo do projeto. Preferencialmente, use o padrão IEEE para citaçoes.

% OBSERVAÇÃO: A seção de referências deve ser gerada automaticamente usando os comandos \cite{} do LaTex. Recomenda-se o uso do padrão IEEE para citações, conforme abaixo.

%\bibliographystyle{IEEEtran}
%\bibliography{seu_arquivo_de_referencias}

\end{document}
