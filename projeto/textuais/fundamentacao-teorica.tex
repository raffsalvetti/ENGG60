% FUNDAMENTAÇÃO TEÓRICA--------------------------------------------------------

\chapter{FUNDAMENTAÇÃO TEÓRICA}
\label{chap:fundamentacao-teorica}

Este capítulo descreve as tecnologias e conceitos centrais utilizados durante a
concepção do projeto. As definições apresentadas são embasadas no material bibliográfico
revisado, que serviu de apoio no desenvolvimento de um trabalho fundamentado nas teorias
existentes.

\section{Raspberry Pi}
\label{sec:raspi}

O Raspberry Pi é uma família de computadores em placa única (SOC em inglês), com o tamanho de um cartão de crédito. Inicialmente seu objetivo era promover o ensino de computação (programação) básica em escolas, principalmente públicas, de todo o mundo. Entretanto, por possuir poder computacional razoável, uma boa quantidade de memória ram (a partir do modelo B) e um preço relativamente baixo, passou a ser usado para outros objetivos como: console de videogame clássico (emulação de jogos), gerencia de mídia (vídeos, fotos e musicas), estudos em eletrônica, domótica (automação residencial), internet das coisas e robótica. \citeonline{jucapereira2018aplicacoes} \par
Uma versão do sistema operacional Debian Linux, chamada Raspbian, foi criada para o Raspberry Pi, portando também uma serie de aplicativos e ferramentas de desenvolvimentos já existentes para computadores da plataforma PC. Dessa modo, o desenvolvimento de programas se torna uma tarefa extremamente simples, já que o hardware é abstraído pelo sistema operacional, e não é necessário conhecimento especifico do hardware do Raspberry Pi (plataforma ARM). As linguagens mais utilizadas para desenvolvimento de software com bibliotecas disponíveis para interação com o hardware são o C/C++ e Python, porém, é possível desenvolver em outras linguagens de programação como o PHP e Java. \citeonline{jucapereira2018aplicacoes} \par

\begin{figure}[H]
	\centering
	\includegraphics[width=0.6\textwidth]{figuras/raspberrypi_model_b.jpg}
	\caption{Raspberry Pi (Modelo B).}
	\fonte{ \citeonline{sparkfun2019}}
	\label{fig:raspi_modelb}
\end{figure}

\section{Servo Motor}
\label{sec:servomotor}

Um servo motor, visto na \autoref{fig:servo_g9}, é um atuador rotatório, ou atuador linear, que permite um controle preciso da posição linear ou angular, velocidade e aceleração de uma carga ligada ao seu eixo. Consiste basicamente em um motor de corrente continua (para o caso particular desse trabalho), acoplado a um sensor (potenciômetro), como ilustrado na \autoref{fig:insideaservo}, para ler sua posição durante o movimento. Normalmente, os motores servos necessitam de um sinal de controle modulado por largura de pulso (PWM em ingles) para operar. \citeonline{petruzella2009electric} \par

\begin{figure}[h]
	\centering
	\includegraphics[width=0.6\textwidth]{figuras/servo_g9.jpg}
	\caption{Servo Micro TG9.}
	\fonte{ \citeonline{hobbyking2019}}
	\label{fig:servo_g9}
\end{figure}

O servo motor opera em malha fechada, isto é, seu controlador compara a velocidade de movimento e sua posição para gerar o próximo comando de movimento, minimizando o erro. \citeonline{petruzella2009electric} O esquema de funcionamento do servo motor é mostrado na figura \autoref{fig:servo_closed_loop}. 

\begin{figure}[h]
	\centering
	\begin{subfigure}{.5\textwidth}
		\includegraphics[width=0.95\textwidth]{figuras/servo_closed_loop.png}
		\caption{Sistema em malha fechada.}
		\fonte{ \citeonline{petruzella2009electric}}
		\label{fig:servo_closed_loop}
	\end{subfigure}%
	\begin{subfigure}{.5\textwidth}
		\includegraphics[width=0.95\textwidth]{figuras/inside_a_servo.jpg}
		\caption{Componentes internos.}
		\fonte{\citeonline{pinckney2006pulse}.}
		\label{fig:insideaservo}
	\end{subfigure}
	\caption{Sistema de controle de um servo motor.}
\end{figure}

\section{Modulação por Largura de Pulso}
\label{sec:pwm}

A modulação por largura de pulso é uma técnica empregada em diversas áreas da eletrônica, sendo utilizada para controlar fontes chaveadas, velocidade de motores, luminosidade, servo motores e diversas outras aplicações. Consiste em variar a o tempo em que um pulso de tensão oscila entre os níveis alto e baixo numa taxa rápida o suficiente para que a média dos pulsos crie um valor médio de tensão efetivo, ilustrado na \autoref{fig:pwm}. Ao valor definido pela divisão entre: a largura de pulso com a tensão em nível alto, e o período do sinal; é dado o nome ciclo de trabalho ou \textit{dutty cycle}. Variar o \textit{dutty cycle} significa variar a tensão média, isto é, a potência é proporcional a tensão média resultante. \citeonline{pinckney2006pulse}.

\begin{figure}[h]
	\centering
	\includegraphics[width=1\textwidth]{figuras/pwm.jpg}
	\caption{Modulação por largura de pulso e tensão média resultante}
	\fonte{ \citeonline{pinckney2006pulse}}
	\label{fig:pwm}
\end{figure}

A largura de pulso pode ser calculada usando a equação \autoref{eq:duttycicle}.

\begin{equation}
{Dutty Cycle} = 100 \times \frac{Largura do Pulso}{Período}  
\label{eq:duttycicle}
\end{equation}

Onde: Duty Cycle é um valor dado em porcento, Largura do pulso e Período são valores dados em segundos.\par

Uma vez calculado o ciclo de trabalho, é possível calcular o valor da tensão média gerada pelo sinal através da \autoref{eq:avervoltage}.

\begin{equation}
{Tensão Média} = {Tensão do Pulso} \times {Duty Cycle}
\label{eq:avervoltage}
\end{equation}

\section{Smartfone Android}
\label{sec:android}

O smartfone é um dispositivo móvel que mescla recursos de um telefone celular (receber e efetuar chamadas, mensagens de texto curto e etc) e recursos de um computador pessoal garantindo a possibilidade de instalar novos aplicativos e assim, agregar novas funcionalidades ao aparelho.\par

O Android é um sistema operacional de código livre, baseado em núcleo Linux (responsável por gerenciar dispositivos de entrada e saída, memória e processos), marcado em vermelho na \autoref{fig:androidsysarch}, e desenvolvido por uma das maiores empresas de tecnologia da atualidade, a Google. \citeonline{ronamadeo2018} Sua interface com o usuário é baseada na manipulação direta, isto é, apresenta continuamente o objeto de interesse permitindo sua manipulação usando recursos que correspondem proximamente ao mundo físico. \citeonline{barbosa2010interaccao} Atualmente o sistema operacional Android pode ser encontrado em outros aparelhos como relógios, televisores e dispositivos gerenciadores de mídia. \citeonline{androidcom2019} \par

\begin{figure}[H]
	\centering
%	\begin{subfigure}{.5\textwidth}
		\includegraphics[width=0.7\textwidth]{figuras/android_sys_arch.pdf}
		\caption{Arquitetura do sistema operacional Android.}
		\fonte{ \citeonline{brady2008anatomy}.}
		\label{fig:androidsysarch}
%	\end{subfigure}%
%	\begin{subfigure}{.5\textwidth}
%		\includegraphics[width=0.95\textwidth]{figuras/axis_globe.png}
%		\caption{em relação ao globo terrestre.}
%		\label{fig:axisglobe}
%	\end{subfigure}
%	\caption{Sistema operacional Android.}
\end{figure}

Boa parte dos smartfones vendidos atualmente com o sistema operacional Android, possuem uma gama de sensores embutidos capazes de fornecer dados, com um grau de precisão aceitável, relativos a orientação e movimento do aparelho e até condições do ambiente como iluminação, pressão atmosférica e umidade do ar. \citeonline{androidcom2019} Esses sensores encontram-se divididos em três grupos básicos na API de desenvolvimento fornecida pelo Android. Sensores \textit{Ambientais}, que podem ser utilizados em aplicações simples como coletar o nível de iluminação do ambiente, a umidade e temperatura para calcular o ponto de orvalho (condição em que a água em vapor presente num ambiente se condensa tornando-se o orvalho), e os sensores de \textit{Posição} e \textit{Movimento}, que podem ser usados em aplicações como jogos que, por exemplo, utilizam a aceleração da gravidade para inferir movimentos complexos do usuário como rotações, sacudidas e inclinações do aparelho. Este último grupo sendo o ponto de interesse para o trabalho, e portanto detalhado adiante.\par

\subsection{Sensores de Movimento}
\label{subsec:motionsensors}

A plataforma de desenvolvimento do Android oferece vários sensores que permitem sentir o movimento de um dispositivo. Dentre eles, os mais utilizados são os sensores de gravidade, aceleração linear e vetor de rotação, que podem ser implementados em hardware ou software (baseando-se em dados fornecidos por outros sensores implementados em hardware), e os sensores acelerômetro e giroscópio, que sempre são implementados em hardware. Todos eles descrevem o movimento do dispositivo em formato matricial.\par

De forma geral, os sensores usam um sistema de coordenadas com três eixos para expressas os dados. Para a maioria dos sensores da API, o sistema de coordenadas é definido relativo a tela do dispositivo, quando segurado na orientação padrão (porta retrato para celulares e paisagem para tablets), como mostrado na \autoref{fig:axisdevice}. Dessa forma, o eixo X é horizontal e aponta para a direita, o eixo Y é vertical e aponta para cima e o eixo Z, perpendicular a tela do dispositivo, estando seus valores positivos do lado da tela e os negativos atrás da tela. Um ponto importante de ser notado, é que a orientação do sistema de eixos não muda quando o dispositivo é segurado de maneira diferente da sua orientação padrão.\par

\begin{figure}[H]
	\centering
	\begin{subfigure}{.5\textwidth}
		\includegraphics[width=0.85\textwidth]{figuras/axis_device.png}
		\caption{em relação ao aparelho.}
		\label{fig:axisdevice}
	\end{subfigure}%
	\begin{subfigure}{.5\textwidth}
		\includegraphics[width=0.95\textwidth]{figuras/axis_globe.png}
		\caption{em relação ao globo terrestre.}
		\label{fig:axisglobe}
	\end{subfigure}
	\caption{Sistema de coordenadas.}
	\fonte{ \citeonline{google2019devsensors}.}
\end{figure}

O sensor vetor de rotação representa a orientação do dispositivo como uma combinação  de um angulo e um eixo no qual o dispositivo rotacionou com um ângulo $\theta$ em torno de um eixo (x, y ou z). \citeonline{google2019devsensors} \par
Os três componentes do vetor re rotação são expressos como:\\

\begin{equation}
\Delta_x = x \times sen(\frac{\theta}{2})
\label{eq:vec_rot_x}
\end{equation}

\begin{equation}
\Delta_y = y \times sen(\frac{\theta}{2})
\label{eq:vec_rot_y}
\end{equation}

\begin{equation}
\Delta_z = z \times sen(\frac{\theta}{2})
\label{eq:vec_rot_z}
\end{equation}

Onde a magnitude do vetor de rotação é expresso por $sen(\frac{\theta}{2})$, e sua direção é a mesma do eixo de rotação.\par

A o sistema de coordenada de referencia é definido como uma base ortogonal direta, como mostrado na \autoref{fig:axisglobe}, e possui as seguintes características:\par
\begin{itemize}
\item O eixo X é definido como o produto vetorial dos eixos Y e Z, é tangente a superfície terrestre no local onde o dispositivo se encontra e aponta aproximadamente para o leste.\par

\item O eixo Y é tangente a superfície terrestre no local onde o dispositivo se encontra e aponta para o norte magnético. \par

\item O eixo Z aponta para o céu e é perpendicular ao solo.
\end{itemize}