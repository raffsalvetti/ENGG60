% RESUMO--------------------------------------------------------------------------------

\begin{resumo}[RESUMO]
\begin{SingleSpacing}

Sistemas robóticos fazem parte da história humana. Seja na vida prática, nos dias atuais, ou em peças teatrais de ficção científica, como exemplo, a obra \emph{Rossumovi Univerzální Roboti}, criada em 1920, onde o termo \textbf{"robô"} foi utilizado pela primeira vez, antes mesmo do primeiro computador eletrônico ser construído.
Atualmente, a robótica é usada nos mais variados setores. É usada como auxiliar de limpeza doméstica nas residências (por exemplo, na forma de pequenos robôs aspiradores de pó), são utilizados como seguranças patrimoniais em alguns países, auxiliam cirurgias na medicina e exploram mundos desconhecidos no espaço. Usando elementos de robótica, este trabalho pretende usar os dados capturados pelos sensores de movimento de um \textit{smartphone}, para criar um sistema de controle de movimento para câmera, operado via rede sem fio, que pode ser embarcada num robô, fornecendo para o seu operador uma interface mais intuitiva. São apresentadas as características do projeto, montagem do protótipo, programação dos softwares de controle de câmera, envio de dados dos sensores, acionamento de motores e resultados.\\

\textbf{Palavras-chave}: Robótica, Movimento, Controle, Automação, Câmera.

\end{SingleSpacing}
\end{resumo}

% OBSERVAÇÕES---------------------------------------------------------------------------
% Altere o texto inserindo o Resumo do seu trabalho.
% Escolha de 3 a 5 palavras ou termos que descrevam bem o seu trabalho .
% As palavras-chave são separadas por pontos. Apenas a primeira letra é maiúscula.

