% RESUMO--------------------------------------------------------------------------------

\begin{resumo}[RESUMO]
\begin{SingleSpacing}

Sistemas robóticos fazem parte da história humana. Seja na vida pratica, nos dias atuais, ou na peça teatral de ficção cientifica \emph{Rossumovi Univerzální Roboti}, criada em 1920, onde o termo "ROBÔ" foi utilizado pela primeira vez, antes mesmo do primeiro computador eletrônico ser construído.
Atualmente, a robótica é usada nos mais variados setores. Nos domicílios é usado como auxiliar de limpeza doméstica (na forma de aspiradores de pó), em alguns países são utilizados como seguranças patrimoniais, na medicina auxiliam em cirurgias e no espaço exploram mundos desconhecidos. Usando elementos de robótica, este trabalho pretende usar dados de sensores de posição de um smartphone, para criar um sistema de controle de movimento para câmera que pode ser embarcada num robô, fornecendo para o operador um controle intuitivo. São apresentados as características do projeto, montagem do protótipo, programação dos softwares de controle de câmera e envio de dados dos sensores.\\

\textbf{Palavras-chave}: Robótica, Movimento, Controle, Automação, Câmera.

\end{SingleSpacing}
\end{resumo}

% OBSERVAÇÕES---------------------------------------------------------------------------
% Altere o texto inserindo o Resumo do seu trabalho.
% Escolha de 3 a 5 palavras ou termos que descrevam bem o seu trabalho .
% As palavras-chave são separadas por pontos. Apenas a primeira letra é maiúscula.

